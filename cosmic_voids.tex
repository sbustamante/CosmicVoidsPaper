\documentclass[a4,useAMS,usenatbib,usegraphicx]{latex/mn2e} 
%\documentclass{latex/emulateapj} 
%External Packages and personalized macros
%=========================================================================
%		EXTERNAL PACKAGES
%=========================================================================
\usepackage{amsmath} 
\usepackage{amssymb} 
%\usepackage[section]{placeins}
\usepackage {graphicx}
%\usepackage{graphics}
\usepackage[dvips]{epsfig}
\usepackage{epsfig}  
\usepackage{color}
\usepackage[normalem]{ulem}
\usepackage{hyperref}
\usepackage{caption}
%Non reposionated tables
\usepackage{float}
\restylefloat{table}

%=========================================================================
%		INTERNAL MACROS
%=========================================================================
\def\be{\begin{equation}}
\def\ee{\end{equation}}
\def\ba{\begin{eqnarray}}
\def\ea{\end{eqnarray}}

% To highlight comments 
\definecolor{red}{rgb}{1,0.0,0.0}
\newcommand{\red}{\color{red}}
\definecolor{darkgreen}{rgb}{0.0,0.5,0.0}
\newcommand{\SRK}[1]{\textcolor{darkgreen}{\bf SRK: \textit{#1}}}
\newcommand{\SRKED}[1]{\textcolor{darkgreen}{\bf #1}}

\newcommand{\LCDM}{$\Lambda$CDM~}
\newcommand{\beq}{\begin{eqnarray}}  
\newcommand{\eeq}{\end{eqnarray}}  
\newcommand{\zz}{$z\sim 3$} 
\newcommand{\apj}{ApJ}  
\newcommand{\apjs}{ApJS}  
\newcommand{\apjl}{ApJL}  
\newcommand{\aj}{AJ}  
\newcommand{\mnras}{MNRAS}  
\newcommand{\mnrassub}{MNRAS accepted}  
\newcommand{\aap}{A\&A}  
\newcommand{\aaps}{A\&AS}  
\newcommand{\araa}{ARA\&A}  
\newcommand{\nat}{Nature}  
\newcommand{\physrep}{PhR}
\newcommand{\pasp}{PASP}    
\newcommand{\pasj}{PASJ}    
\newcommand{\avg}[1]{\langle{#1}\rangle}  
\newcommand{\ly}{{\ifmmode{{\rm Ly}\alpha}\else{Ly$\alpha$}\fi}}
\newcommand{\hMpc}{{\ifmmode{h^{-1}{\rm Mpc}}\else{$h^{-1}$Mpc}\fi}}  
\newcommand{\hGpc}{{\ifmmode{h^{-1}{\rm Gpc}}\else{$h^{-1}$Gpc}\fi}}  
\newcommand{\hmpc}{{\ifmmode{h^{-1}{\rm Mpc}}\else{$h^{-1}$Mpc}\fi}}  
\newcommand{\hkpc}{{\ifmmode{h^{-1}{\rm kpc}}\else{$h^{-1}$kpc}\fi}}  
\newcommand{\hMsun}{{\ifmmode{h^{-1}{\rm {M_{\odot}}}}\else{$h^{-1}{\rm{M_{\odot}}}$}\fi}}  
\newcommand{\hmsun}{{\ifmmode{h^{-1}{\rm {M_{\odot}}}}\else{$h^{-1}{\rm{M_{\odot}}}$}\fi}}  
\newcommand{\Msun}{{\ifmmode{{\rm {M_{\odot}}}}\else{${\rm{M_{\odot}}}$}\fi}}  
\newcommand{\msun}{{\ifmmode{{\rm {M_{\odot}}}}\else{${\rm{M_{\odot}}}$}\fi}}  
\newcommand{\lya}{{Lyman$\alpha$~}}
\newcommand{\clara}{{\texttt{CLARA}}~}
\newcommand{\rand}{{\ifmmode{{\mathcal{R}}}\else{${\mathcal{R}}$ }\fi}}  
%SAMPLES
\newcommand{\GHBDM}{\texttt{GH}$_{\mbox{\tiny{BDM}}}$ }
\newcommand{\GHFOF}{\texttt{GH}$_{\mbox{\tiny{FOF}}}$ }
\newcommand{\IHBDM}{\texttt{IH}$_{\mbox{\tiny{BDM}}}$ }
\newcommand{\IHFOF}{\texttt{IH}$_{\mbox{\tiny{FOF}}}$ }
\newcommand{\PBDM}{\texttt{P}$_{\mbox{\tiny{BDM}}}$ }
\newcommand{\PFOF}{\texttt{P}$_{\mbox{\tiny{FOF}}}$ }
\newcommand{\IPBDM}{\texttt{IP}$_{\mbox{\tiny{BDM}}}$ }
\newcommand{\IPFOF}{\texttt{IP}$_{\mbox{\tiny{FOF}}}$ }
\newcommand{\RIPBDM}{\texttt{RIP}$_{\mbox{\tiny{BDM}}}$ }
\newcommand{\RIPFOF}{\texttt{RIP}$_{\mbox{\tiny{FOF}}}$ }


%MY COMMANDS #############################################################
\newcommand{\sub}[1]{\mbox{\scriptsize{#1}}}
\newcommand{\dtot}[2]{ \frac{ d #1 }{d #2} }
\newcommand{\dpar}[2]{ \frac{ \partial #1 }{\partial #2} }
\newcommand{\pr}[1]{ \left( #1 \right) }
\newcommand{\corc}[1]{ \left[ #1 \right] }
\newcommand{\lla}[1]{ \left\{ #1 \right\} }
\newcommand{\bds}[1]{\boldsymbol{ #1 }}
\newcommand{\oiint}{\displaystyle\bigcirc\!\!\!\!\!\!\!\!\int\!\!\!\!\!\int}
\newcommand{\mathsize}[2]{\mbox{\fontsize{#1}{#1}\selectfont $#2$}}
\newcommand{\eq}[2]{\begin{equation} \label{eq:#1} #2 \end{equation}}
\newcommand{\lth}{$\lambda_{th}$ }
%#########################################################################

\begin{document}

%=========================================================================
%		FRONT MATTER
%=========================================================================
\title{Fractional anisotropy as a tracer of cosmic voids}
\author[S. Bustamante and J.E. Forero-Romero]{
\parbox[t]{\textwidth}{\raggedright 
  Sebastian Bustamante \thanks{sebastian.bustamante@udea.edu.co}$^{1}$,
  Jaime E. Forero-Romero$^{2}$ 
}
\vspace*{6pt}\\
$^1$Instituto de F\'{\i}sica - FCEN, Universidad de Antioquia, Calle
67 No. 53-108, Medell\'{\i}n, Colombia\\ 
$^2$Departamento de F\'{i}sica, Universidad de los Andes, Cra. 1
No. 18A-10, Edificio Ip, Bogot\'a, Colombia
}

\maketitle

\begin{abstract}
Finding and characterizing underdense regions (voids) in the large
scale structure of the Universe is an important task in cosmological
studies.  
In this paper we present a novel approach to find voids in
cosmological simulations.  
Our approach is based on algorithms that use the tidal and the
velocity shear tensors to locally define the cosmic web.
Voids are identified using the fractional anisotropy (FA) computed
from the eigenvalues of each web scheme. 
We define the void boundaries using a watershed transform based on the
local minima of the FA and its boundaries as the regions where the FA
is maximized
This void identification technique does not have any free parameters
and does not make any assumption on the shape or structure of the
voids.  
We test the method on the Bolshoi simulation and report on the density
and velo ity profiles for the voids found using this new scheme. 
\end{abstract}

\begin{keywords}
Cosmology: theory - large-scale structure of Universe -
Methods: data analysis - numerical - N-body simulations
\end{keywords}


%=========================================================================
%		PAPER CONTENT
%=========================================================================

%*************************************************************************
\section{Introduction}
\label{sec:introduction}
%*************************************************************************


Since voids were found in the first compiled galaxy surveys 
they have been identified as one of the most striking features of the
Cosmic Web \citep{Chincarini75,  Gregory78, Einasto80M, Einasto80N,
  Kirshner81,  Kirshner87, Bond96}. 
However,  due to the large volume extension of void regions ($\sim
5-10\ \mbox{Mpc}  h^{-1}$), statistically meaningful catalogues of
voids \citep{Pan10,  Sutter12b, Nadathur14} have only become available
through  modern galaxy surveys like the two-degree field Galaxy
Redshift Survey \citep{ Colless01, Colless03} and the Sloan Digital
Sky Survey \citep{York00, Abazajian03}.
This advancement generated a great interest to study voids
observationally during the last decade \citep{Hoyle04, Croton04, Rojas05,
  Ceccarelli06, Patiri06a, Tikhonov06, Patiri06b,Tikhonov07,
  BendaBeckmann08, Foster09, Ceccarelli13, Sutter14a}. 


On the theoretical side, early descriptions of the evolution of the 
large-scale Universe, based on gravitational instabilities in primordial 
stages and leaded by the seminal work of \citet{Zeldovich70}, are 
consistent with the Cosmic Web picture, where planar pancake-like regions 
of matter enclose enormous underdense voids and are bordered, in turn, by 
thin filaments and high-density clumpy knots \citep{Bond96}. First 
theoretical models for describing formation, dynamics and properties of 
voids \citep{Hoffman82, Icke84, Bertschinger85, Blumenthal92} were 
dramatically complemented and extended by first numerical studies based 
on simulations \citep{Martel90, Regos91, Weygaert93, Dubinski93}. This 
tendency of using numerical data from N-body simulations, fuelled by last 
generation computing systems and ever more efficient numerical algorithms, 
has become increasingly common in the last years as a powerful analysis 
toolkit of cosmic voids (for an extensive compilation of previous 
numerical works, see \citet{Colberg08}).


At present, studying voids may be considered as a threefold enterprise
\citep{Platen07}: first, they are a key ingredient of the Cosmic Web as 
they dominate almost the entire volume distribution at large-scales and 
additionally, they compensate overdense structures in the total budget of 
matter. This implies that a full understanding of their evolution and 
properties is essential for fathoming the high complexity of the Cosmic 
Web. Second, they provide a valuable resource for inferring and probing 
cosmological parameters as their structure and dynamics are highly 
determined by these values. Third, they constitute an unique and still 
largely pristine environment in which can be tested galaxy evolution.


Although a visual recognition of voids in galaxy surveys and simulations
is possible in most cases, a formal systematic identification is necessary 
for statistically meaningful studies. However, a basic yet essential issue 
remains regarding the definition itself of what a void is. There is not a 
general consensus about an unambiguous definition of cosmic voids and 
therefore, there can be found many different void finding techniques 
throughout the literature (for a detailed comparison of different schemes, 
see the Void Finder Comparison Project \citet{Colberg08}). In spite of the 
diversity of existing schemes, they can be roughly classified into two 
main types: first, geometric schemes based on point spatial or redshift 
distribution of galaxies in surveys or catalogues of dark matter halos in 
simulations \citep{Kauffmann91, Muller00, Gottlober03, Hoyle04, Brunino07, 
Foster09, Micheletti14, Sutter14}, and second, schemes based on the smooth 
and continuous matter density field either from simulations or from 
reconstruction procedures on surveys \citep{Plionis02, Colberg05, 
Shandarin06, Platen07, Neyrinck08, Neyrinck13, Ricciardelli2013}.


Here we introduce a new tracer of the structure of cosmic voids built from 
two tensorial web schemes, i.e. the Tweb, based on the Hessian of the 
gravitational potential or tidal tensor \citep{Hahn07, Forero09}, and the 
Vweb, based on the velocity shear tensor \citep{Hoffman12}. These web 
schemes classify the Cosmic Web into four different types of environment
depending on the counting of the number of eigenvalues below an 
user-defined threshold ($\lambda_{th}$), namely voids (3 eigenvalues below
$\lambda_{th}$), sheets (2), filaments (1) and knots or clusters (0).
The tidal and the shear tensors have proven to be more fundamental than 
the density field as they also trace the collapsing or expanding nature of 
the matter field, which ultimately is what defines the underlying dynamics 
of the Cosmic Web. Furthermore, the density field is degenerated regarding 
the defined types of environment \citep{Hahn07}, which implies that is not 
possible to assign a priori a range of density values to each of them. 
This yields that the usually adopted definition of voids as simply 
underdense regions in the large-scale matter distribution, is not precise 
enough as it excludes a proper description of the internal structure of 
voids.


Following the recent trend of using digital image-processing techniques 
developed in other disciplines (e.g. medical sciences and computer imaging) 
for analysing the structure of the Cosmic Web, we propose here, much in 
the same way as \citet{Libeskind13}, the fractional anisotropy (FA) as a 
tracer of the internal structure and the outline of cosmic voids. The FA 
was initially introduced by \citet{Basser95} for quantifying the 
anisotropy degree of the diffusivity of water molecules through cerebral
tissue in nuclear magnetic resonance imaging. In this context we use the 
same quantity for determining the anisotropy degree of the local 
environment from either the orbital dynamics as set by the eigenvalues of
the tidal tensor (Tweb), or the dynamics of the velocity field as set by 
the eigenvalues of the shear tensor (Vweb). In both cases, the FA is not
determined directly from the density field, it hence is suitable for 
describing both, the structure of high non-linear regions (e.g. filaments,
knots and very dense walls) as well as the fainter substructure exhibited 
by quasi-linear regions like voids.


Once established the FA as the tracer field of cosmic voids, we proceed to
identify individual voids as basins of local minima. For this purpose we
implement a technique developed by \citet{Beucher79} and \citet{Beucher93} 
and widely used in the field of image analysis and mathematical morphology, 
i.e. the \textit{watershed transform algorithm}. This technique has been 
already implemented on void finding schemes by \citet{Platen07} and 
\citet{Neyrinck08} with very interesting results, where individual voids 
are identified as catching basins of local minima of the density field. 
The appeal of this algorithm consists in that is parameter-free and does 
not require any assumption on the shape and morphology of voids. 
Although we use a \textit{cloud-in-cell} (\texttt{CIC}) algorithm on a 
Cartesian mesh for estimating the density and tensor fields, instead of 
the more sophisticated \textit{Delaunay tessellation for field estimator} 
(\texttt{DTFE}) technique \citep{Schaap00}, our implementation of the 
watershed transform should not be significantly affected as we are 
interested in low density regions where the \texttt{CIC} gives similar 
estimations.


This paper is organized as follows. In section 2 we describe the used
simulation for testing our void finding algorithm, i.e. the Bolshoi
simulation. A detailed description of these methods is presented in 
section 3. In section 4 we describe how the FA can be used for tracing the 
structure of voids. Once obtained meaningful catalogues of voids for each 
scheme, we proceed to compare some typical and general interest properties 
of voids such as volume functions, distributions of size and density and 
velocity profiles. This is done in section 5. Finally in section 6 we 
analyse and evaluate our findings.



%*************************************************************************
\section{The Simulation}
\label{sec:the_simulation}
%*************************************************************************


We use here an unconstrained cosmological simulation (the Bolshoi 
simulation) to identify the possible large scale environment and the 
distribution of cosmic voids at $z=0$. The Bolshoi simulation follows the 
non-linear evolution of a dark matter density field on a cubic volume of 
size $250$\hMpc sampled with $2048^3$ particles. The cosmological 
parameters in the simulation are $\Omega_{\rm m}=0.27$, $\Omega_{\Lambda}
=0.73$, $h=0.70$, $n=0.95$ and $\sigma_{8}=0.82$ for the matter density, 
cosmological constant, dimensionless Hubble parameter, spectral index of 
primordial density perturbations and normalization for the power spectrum
respectively, consistent with the seventh year of data of the Wilkinson 
Microwave Anisotropy Probe (WMAP) \citep{Jarosik11}. For more detailed 
technical information about the simulation, see \citet{Klypin11}.


For estimating the density and velocity fields we use a 
\textit{cloud-in-cell} (\texttt{CIC}) algorithm onto a grid of $256^3$ 
cells, corresponding to a resolution of $0.98$\hMpc per cell side. Then, 
through finite-differences and FFT methods the tidal and shear tensors 
are computed. Finally, the eigenvalues and eigenvectors of the tensor are
obtained for each cell of the grid. Neglecting substructures presented 
below Megaparsec scales and taking into account our focus in voids, which
are a prominent characteristic of the Megaparsec Universe, we apply a 
Gaussian softening of one cell to all fields.



%*************************************************************************
\section{Algorithms to quantify the Cosmic Web}
\label{sec:algorithms_cosmic_web}
%*************************************************************************


%-------------------------------------------------------------------------
\subsection{The tidal web (Tweb)}
\label{subsec:Tweb}
%-------------------------------------------------------------------------


This scheme was initially proposed by \citet{Hahn07} as a novel 
alternative for classifying the Cosmic Web based on the tidal tensor, 
that is somehow more fundamental than the density field as it also allows 
to quantify the orbital dynamics of the matter field. This approach 
consists of a second-order expansion of the equations of motion around 
local minima of the gravitational potential and then extended to any 
position. The second-order term corresponds to the tidal tensor, which is
defined as the Hessian matrix of the normalized gravitational potential


%.........................................................................
%Tidal Tensor
\eq{V_web}
{	T_{\alpha\beta} = \frac{\partial^2\phi}{\partial x_{\alpha}\partial x_{\beta}}	}
%.........................................................................
where the physical gravitational potential has been rescaled by a factor 
$4\pi G\bar{\rho}$ in such a way that $\phi$ satisfies the following 
equation


%.........................................................................
%Poisson
\eq{Poisson}
{	\nabla^2\phi = \delta,	}
%.........................................................................
with $\bar{\rho}$ the average density in the Universe, $G$ the 
gravitational constant and $\delta$ the dimensionless matter overdensity.


Since the tidal tensor can be represented in any cell by a real and 
symmetric $3\times 3$ matrix, it is ensured the possibility to diagonalize 
it and obtain three real eigenvalues $\lambda_{1}\geq\lambda_{2}\geq
\lambda_3$. This set of eigenvalues can be used as indicators of the local 
orbital stability in each proper direction, which in turn can be 
translated into a classification scheme of the Cosmic Web. A counting of 
the number of positive (stable) or negative (unstable) eigenvalues allows 
to catalogue a single cell into one of the next four types of environment: 
voids (3 negatives eigenvalues), sheets (2), filaments (1) and knots (0). 
A significant improvement to this scheme was introduced by \citet{Forero09}
by means of a relaxation of the stability criterion. The relative strength 
of each eigenvalue is no longer defined by the sign, but instead by a
threshold value $\lambda_{th}$ that can be tuned in such a way that the
visual impression of the web-like matter distribution is reproduced.



%-------------------------------------------------------------------------
\subsection{The velocity web (Vweb)}
\label{subsec:Vweb}
%-------------------------------------------------------------------------


We also use a kinematical scheme to define the Cosmic Web environment in 
the simulation. The scheme has been thoroughly described in 
\citet{Hoffman12} and applied to study the shape and spin alignment in the
Bolshoi simulation in \citet{Libeskind13}. We refer the reader to these 
papers to find a detailed description of the algorithm, its limitations 
and capabilities. The Vweb scheme for environment finding is based on the
local velocity shear tensor calculated from the smoothed dark matter 
velocity field in the simulation. The central quantity is given by the 
following dimensionless expression


%.........................................................................
%V-Web Definition
\eq{V_web}
{	\Sigma_{\alpha\beta} = -\frac{1}{2H_0}\pr{\frac{\partial v_{\alpha}}
{\partial x_{\beta}}+\frac{\partial v_{\beta}}{\partial x_{\alpha}}}	}
%.........................................................................
where $v_{\alpha}$ and $x_{\alpha}$ represent the $\alpha$ component of 
the comoving velocity and position, respectively. Like the tidal tensor, 
$\Sigma_{\alpha\beta}$ can be represented by a $3\times 3$ symmetric 
matrix with real values, hence diagonalizing it is obtained three real 
eigenvalues $\lambda_{1}\geq\lambda_{2}\geq\lambda_3$ whose sum (the 
trace of $\Sigma_{\alpha\beta}$) is proportional to the divergence of the 
local velocity field smoothed on the physical scale ${\mathcal R}$. 


In the same way, the relative strength of the three eigenvalues with 
respect to a threshold value $\lambda_{th}$ allows for the local 
classification of the matter distribution into the previous four web types. 
For the threshold choosing in both schemes, the Tweb and the Vweb, it is
usual to fine-tuning the value in such a way that the visual appearance of
the Cosmic Web as seen in simulations and galaxy surveys is reproduced. 
However, we do not take this approach here, instead we propose a novel 
approach for the threshold choosing based on the maximization of the 
fractional anisotropy field occurring in filaments and very dense walls.


%*************************************************************************
\section{Finding voids}
\label{sec:bulk_voids}
%*************************************************************************


%-------------------------------------------------------------------------
\subsection{The fractional anisotropy}
\label{subsec:FA_voids}
%-------------------------------------------------------------------------


The fractional anisotropy (FA), as developed by \citet{Basser95}, was 
conceived to quantify the anisotropy degree of a diffusion process, e.g. 
the diffusivity of water molecules through cerebral issue in nuclear
magnetic resonance imaging. Here we propose the FA, much in the same 
way as \citet{Libeskind13}, as a tracer of cosmic voids.


%.........................................................................
%Fractional anisotropy
\eq{fractional_anisotropy}
{ FA = \frac{1}{\sqrt{3}}\sqrt{ \frac{ (\lambda_1 - \lambda_3)^2 + 
(\lambda_2 - \lambda_3)^2 + (\lambda_1 - \lambda_2)^2}{ \lambda_1^2 + 
\lambda_2^2 + \lambda_3^2} } }
%.........................................................................
where the eigenvalues are taken from either the Tweb or the Vweb 
(FA-Tweb and FA-Vweb respectively). Such as it is defined, $FA=0$ 
corresponds with an isotropic dynamic and $FA=1$ with a highly anisotropic 
distribution.


In left panels of Fig. \ref{fig:FA_field} we calculate the FA for both 
web schemes over a slide of the simulation. Some important points can be 
concluded from this figure:


%.........................................................................
\begin{itemize}
\item Voids display a highly isotropic expanding dynamic at their centres, 
becoming gradually more anisotropic at outer regions.
\item Knots feature with very isotropic collapses. For the Tweb, the FA 
exhibits very narrow distribution around knots. For the Vweb, these
distributions are more spread out, thereby indicating strong differences 
between the density and the velocity fields in highly non-linear regions.
\item The filamentary structure of the Cosmic Web is well traced by high 
FA values (black regions), thus indicating very anisotropic dynamics for 
sheets and filaments.
\item The FA, unlike the density field, displays a non-monotonic behaviour, 
where low values are characteristic of central regions of voids, reaching 
high values in sheets and filaments and becoming low again in knots. 
%\item The density field poorly traces the Cosmic Web. This is a somehow 
%expected result due to the semi-infinite range of the density field, what 
%makes more difficult to trace structures.
\end{itemize}
%.........................................................................


%-------------------------------------------------------------------------
\subsection{Voids using web schemes}
\label{subsec:web_voids}
%-------------------------------------------------------------------------


For both web schemes, voids are regions where $\lambda_3\leq\lambda_2\leq
\lambda_1\leq\lambda_{th}$. This implies that the outlines of voids
are completely fixed by the relative strength of the $\lambda_1$ 
eigenvalue with respect to the threshold value. Therefore, as we 
increase/decrease the threshold value $\lambda_{th}$, voids expand/diminish
progressively through contours of $\lambda_1$. In Fig. 
\ref{fig:L1_correlations} we calculate the distribution of the FA as well
as of the density field with respect to $\lambda_1$ over all cells of the 
grid.


From Fig. \ref{fig:L1_correlations} we conclude that the FA is a good
tracer of voids as is almost perfectly correlated with low values of 
$\lambda_1$. Then, it reaches a maximum value, namely sheets and filaments, 
for finally reaching knots, which feature low FA values. This behaviour can 
be thought as a sort of one-dimensional tomography of the Cosmic Web. This
characteristic allows us to propose an optimal threshold value for both 
web schemes where the FA is maximized. Specifically we propose a value of 
$FA=0.95$, corresponding with a threshold $\lambda_{opt}^T = 0.265$ for
the Tweb and $\lambda_{opt}^V = 0.175$ for the Vweb.  For a threshold 
above these values, voids would span over very anisotropic regions, which 
must correspond to sheets and filaments. A threshold below would imply 
very low volume filling fractions of voids and dominant sheets. Finally, 
the density field do not correlate well with $\lambda_1$ (as calculated 
for both web schemes), and due to its monotonic increase, an estimate of 
an optimal threshold is not as clear as for the FA.


Once set the optimal thresholds for the web schemes, we proceed in Fig. 
\ref{fig:FAShapeness} to sample the FA and the prolateness for a random 
sample of cells previously classified into one of the four environments. 
The prolateness is introduced here for differentiating sheets and filaments
featuring high FA values. In order to illustrate the underlying local 
dynamics sampled by the cells, we also sketch different spheroidal 
geometries according to the relative strength of each eigenvalue, and 
then, we associate them to different ranges of the FA and the prolateness. 
From this, we conclude sheets display very anisotropic distributions, 
namely above $FA=0.95$. They are also biased towards oblate geometries, 
however there are some elongated sheets as well. Filaments are not as 
anisotropic as sheets, ranging from middle up to high FA values. They 
exhibit prolate geometries, but a considerable fraction of them are biased 
towards slightly more oblate values. Voids and knots are the only 
environments featuring with low FA values, thus indicating very isotropic 
expanding/collapsing dynamics at their centres. However, voids span over 
a very wide range of FA and geometries, whereas knots only span over low 
to middle FA values and display a biased oblate geometry.


%-------------------------------------------------------------------------
\subsection{Identifying void regions through web schemes}
\label{subsec:identification}
%-------------------------------------------------------------------------




%*************************************************************************
\section{Properties of voids}
\label{sec:properties}
%*************************************************************************


Once defined our method to classify bulk voids based upon web 
classification schemes of the cosmic web, we proceed to analyse some 
physical properties in order to compare their consistency with the 
geometry of voids as quantified by our method and by density-based schemes.
Next, through the reduced inertia tensor we quantify the shape distribution 
of voids. Finally, we compute numerical radial profiles of density and 
peculiar velocity of bulk voids.


%-------------------------------------------------------------------------
\subsection{Statistics of halos in voids}
\label{subsec:shape_voids}
%-------------------------------------------------------------------------


One of the main challenges in observational void finding is the discrete 
nature of galaxy surveys

 we calculate contours of discrete fields like the median mass and 
the local number of local dark matter halos and

, like the inertia values,
the density and peculiar velocities profiles as calculated over the grid 
and profiles of number of halos.


%-------------------------------------------------------------------------
\subsection{Density profile of voids}
\label{subsec:density_voids}
%-------------------------------------------------------------------------


Describing the density profiles of voids is quite important in order to 
compare and match simulation with observational surveys, allowing possible
constrains for different cosmology models \SRKED{[Hamaous, et.al 2014]}. 
Here, and taking into account the previous results, we rather use an 
ellipsoidal approximation to describe and fit the shape of bulk voids, so 
we use the next ellipsoidal radial coordinate to describe density profiles.


%.........................................................................
%Ellipsoidal radial coordinate
\eq{radial_coordinate}
{
r^2 = \frac{x^2}{\tau_1^2} + \frac{y^2}{\tau_2^2} + \frac{z^2}{\tau_3^2},
\ \ \ \ 0\leq r \leq 1
}
%.........................................................................
where we take the principal moments of inertia $\{\tau_i \}$ as the 
lengths of the principal axes of the ellipsoid and each one of the 
cartesian coordinates as measured in the rotated frame of each void.


We use the same analytic density profile that \SRKED{[Hamaous, et.al 2014]} 
to fit the numerical density profiles of our voids.


%.........................................................................
%Density profile
\eq{density_profile}
{
\delta_v(r) = \delta_c\frac{1-(r/r_s)^\alpha}{1+(r/r_v)^\beta}
}
%.........................................................................


\[ \mathcal{C} = \frac{M_v}{\overline{M}} = \frac{3}{2R^{3}} \int_0^{R} [\delta(r) + 1] r^2 dr \]

\[ \mathcal{C} < 1 \]

\[ \mathcal{C} > 1 \]


%*************************************************************************
\section{Conclusions}
\label{sec:conclusions}
%*************************************************************************


%*************************************************************************
\section*{Acknowledgments}  
%*************************************************************************


\bibliographystyle{latex/mn2e}
\bibliography{references}

%.........................................................................
%FIGURE 1: FA and vissual impression
\begin{figure*}
\centering

  \includegraphics[trim = 16mm 10mm 5mm 12mm, clip, keepaspectratio=true,
  width=0.73\textheight]{./figures/cosmicweb_FA_Tweb.pdf}
  \vspace{0.1 cm}
  \includegraphics[trim = 16mm 10mm 5mm 12mm, clip, keepaspectratio=true,
  width=0.73\textheight]{./figures/cosmicweb_FA_Vweb.pdf}
  
  \captionof{figure}{\small Left panels show the visual impression of the 
  FA field over a slide of the simulation for each web scheme (T-web, upper 
  panels. V-web, middle panels). Black regions correspond with high values, 
  i.e. FA$\approx 1$, while white and light yellow regions to FA$\approx 0$. 
  It is worth noting the degeneration of low values of FA for knots and 
  central regions of voids, thus indicating a high isotropy for both 
  processes. In the same way, high values of FA are consistent with the 
  anisotropic geometry exhibited by filaments and very flat sheets. Middle 
  panels sketch the components of the Cosmic Web using the optimal threshold 
  values. Voids corresponds with white, sheets to gray, filaments to dark 
  gray and knots with black regions. Finally right panels sketch the 
  distribution of the catalogued voids by our method.}

  \label{fig:FA_field}

  
\end{figure*}
%.........................................................................


%.........................................................................
%FIGURE 2: Distributions of FA and density regarding the Lambda_1 eigenvalue
\begin{flushleft}
\begin{figure*}
\centering

  \includegraphics[trim = 2mm 2mm 5mm 10mm, clip, keepaspectratio=true,
  width=0.3\textheight]{./figures/FA_L1.pdf}
  \includegraphics[trim = 2mm 2mm 5mm 10mm, clip, keepaspectratio=true,
  width=0.3\textheight]{./figures/delta_L1.pdf}  
  
  \captionof{figure}{\small Distributions of the FA (left panel) and the 
  density field (right panel) with respect to the eigenvalue $\lambda_1$ 
  for each web scheme (Tweb, red lines. Vweb, blue lines) as calculated 
  over all cells of the grid. Thick central lines correspond with the median 
  and filled regions with the $50\%$ of the distribution.}

  \label{fig:L1_correlations}
  \vspace{0.1 cm}

\end{figure*}
\end{flushleft}
%.........................................................................


%.........................................................................
%FIGURE 3: FA of some regions and shape illustration
\begin{flushleft}
\begin{figure*}
\centering

  \includegraphics[trim = 0mm 1mm 0mm 1mm, clip, keepaspectratio=true,
  width=0.3\textheight]{./figures/FA_Prolatenes_Vweb.pdf}
  \includegraphics[trim = 0mm 1mm 0mm 1mm, clip, keepaspectratio=true,
  width=0.3\textheight]{./figures/FA_Prolatenes.png}  
  
  \captionof{figure}{\small Fractional anisotropy and prolateness for a 
  random sample of cells (left panel) and for different spherical 
  geometries (right panel). For the cells, each environment is coded with
  a different colour, i.e. voids corresponds with dark blue points, 
  sheets with green, filaments with orange and knots with cyan. The axis 
  of the spheroids are computed from the relative strengths of the 
  eigenvalues, so their shape sketch the relative expanding/collapsing 
  dynamics into each direction.}

  \label{fig:FAShapeness}
  \vspace{0.1 cm}

\end{figure*}
\end{flushleft}
%.........................................................................


%.........................................................................
%FIGURE 4: volume functions
\begin{flushleft}
\begin{figure*}
\centering

  \includegraphics[trim = 0mm 0mm 0mm 0mm, clip, keepaspectratio=true,
  width=0.3\textheight]{./figures/voids_regions_volume_all.pdf}
  
  \captionof{figure}{\small Volume functions of voids for both built 
  catalogues. Watershed-Tweb (red curves), Watershed-Vweb (blue curves).
  Continuous lines corresponds with the total number of voids, dot-dashed
  with sub-compensated voids and dashed lines with over-compensated voids. }

  \label{fig:volume_function}
  \vspace{0.1 cm}

\end{figure*}
\end{flushleft}
%.........................................................................


%.........................................................................
%FIGURE 5: Density profile of voids for each defined scheme
\begin{flushleft}
\begin{figure*}
\centering
  \includegraphics[trim = 1mm 0mm 5mm 0mm, clip, keepaspectratio=true,
  width=0.24\textheight]{./figures/voids_density_TwebFAG0.pdf}
  \includegraphics[trim = 1mm 0mm 5mm 0mm, clip, keepaspectratio=true,
  width=0.24\textheight]{./figures/voids_velocity_TwebFAG0.pdf}
  \includegraphics[trim = 1mm 0mm 5mm 0mm, clip, keepaspectratio=true,
  width=0.24\textheight]{./figures/voids_FA_TwebFAG0.pdf}
  
  \includegraphics[trim = 1mm 0mm 5mm 0mm, clip, keepaspectratio=true,
  width=0.24\textheight]{./figures/voids_density_VwebFAG0.pdf}
  \includegraphics[trim = 1mm 0mm 5mm 0mm, clip, keepaspectratio=true,
  width=0.24\textheight]{./figures/voids_velocity_VwebFAG0.pdf}
  \includegraphics[trim = 1mm 0mm 5mm 0mm, clip, keepaspectratio=true,
  width=0.24\textheight]{./figures/voids_FA_VwebFAG0.pdf}
  
  \includegraphics[trim = 1mm 0mm 5mm 0mm, clip, keepaspectratio=true,
  width=0.24\textheight]{./figures/voids_density_TwebFAG1.pdf}
  \includegraphics[trim = 1mm 0mm 5mm 0mm, clip, keepaspectratio=true,
  width=0.24\textheight]{./figures/voids_velocity_TwebFAG1.pdf}
  \includegraphics[trim = 1mm 0mm 5mm 0mm, clip, keepaspectratio=true,
  width=0.24\textheight]{./figures/voids_FA_TwebFAG1.pdf}

  
  \includegraphics[trim = 1mm 0mm 5mm 0mm, clip, keepaspectratio=true,
  width=0.24\textheight]{./figures/voids_density_VwebFAG1.pdf}
  \includegraphics[trim = 1mm 0mm 5mm 0mm, clip, keepaspectratio=true,
  width=0.24\textheight]{./figures/voids_velocity_VwebFAG1.pdf}
  \includegraphics[trim = 1mm 0mm 5mm 0mm, clip, keepaspectratio=true,
  width=0.24\textheight]{./figures/voids_FA_VwebFAG1.pdf}  
  

  \captionof{figure}{\small Density of voids for each finding scheme.}

  \label{fig:RhoVel}
  \vspace{0.1 cm}

\end{figure*}
\end{flushleft}
%.........................................................................


\end{document}
